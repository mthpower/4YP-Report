% Chapter 1

\chapter{Introduction} % Chapter title

\label{ch:introduction} % For referencing the chapter elsewhere, use \autoref{ch:introduction} 

%----------------------------------------------------------------------------------------

\section{The INTEGRAL Satellite}
INTEGRAL, the International Gamma Ray Astrophysics Laboratory, is a satellite launched in 2002. Containing four primary instruments, INTEGRAL looks at sources in energy ranges approximately between 5 keV and 10 MeV, from hard X-rays to Gamma rays.

The mission was launched in 2002 by ESA, with support from NASA and the Russian Space Agency. The satellite was placed into a highly elliptical orbit around the Earth, with an orbital period 72 hours. This allows the instruments significant time making observations above the Earth's radiation belts, which would otherwise contribute to the noise of the signal.

\subsection{Coded Masks}
X-rays cannot be focused using conventional optical lenses, so the four principle instruments onboard INTEGRAL all make use of coded masks to form an image. Coded masks use a pattern of transparent and opaque elements in front of the sensor, causing photons in the field of view of the instrument to cast a shadow on the sensor. The design of coded masks is such that a source in the instrument's field of view will cast a unique shadow based on the orientation. Multiple sources can then be observed; each image of each source is separated by performing mathematical algorithms on the sensor output, a process called deconvolution. 

\subsection{IBIS/ISGRI}
IBIS, the Imager on-board the INTEGRAL Satellite, provides the data used in this project. The instrument has two sets of detecting elements placed one above the other, the first to detect hard X-rays and low energy gamma photons, the second detects higher energy gamma photons. The sources selected in this project are observed in the hard X-ray band, which utilises the lower range of frequency sensitivity of IBIS from the first detector, ISGRI, INTEGRAL Soft Gamma Ray Imager.