% Chapter 1

\chapter{Introduction} % Chapter title

\label{ch:introduction} % For referencing the chapter elsewhere, use \autoref{ch:introduction} 

%----------------------------------------------------------------------------------------

\section{The INTEGRAL Satellite}
INTEGRAL, the International Gamma Ray Astrophysics Laboratory, is a satellite launched in 2002. Containing four primary instruments, INTEGRAL looks at sources in energy ranges approximately between 5 keV and 10 MeV, from hard X-rays to Gamma rays. The mission was launched by ESA, with support from NASA and the Russian Space Agency. The satellite was placed into a highly elliptical orbit around the Earth, with an orbital period 72 hours. This allows the instruments significant time making observations above the Earth\'s radiation belts, which would otherwise contribute to the noise of the signal. The satellite has been in operation for over ten years, providing some of the best observations available to astrophysicists\cite{lamport94}.

\section{IBIS/ISGRI}
IBIS, the Imager on-board the INTEGRAL Satellite, provides the majority of the data used in this project. The instrument has two sets of detecting tiles placed one above the other, the first to detect hard X-rays and low energy gamma photons, the second to detect higher energy gamma photons. This project focussed on observations in X-rays, which utilises the lower range of light frequency sensitivity of IBIS from the first detector, ISGRI, INTEGRAL Soft Gamma Ray Imager.

The designs of ISGRI and PICsIT, its higher energy counterpart, are very similar. ISGRI uses an array of Cadmium Telluride tiles in a 128 by 128 array, and PICsIT has an array of Caesium Iodide tiles arranged 64 by 64. Using this dual detector design, IBIS is sensitive between 20 keV to 10 MeV. One additional benefit of this setup is that photons which register in both layers can be identified, which allows an improvement in the signal to noise ratio by discounting those unlikely to have come from a target source.

\section{Other Instruments on-board INTEGRAL}
The Spectrometer on INTEGRAL, abbreviated SPI, is used to provide information on the energy incoming photons that IBIS cannot. SPI has an energy resolution of 2.2 keV at 1.33 MeV, approximately 0.2\%, whilst INTEGRAL has a spectral resolution of between 8\% to 10\%. SPI uses a hexagonal arrangement of 19 germanium tiles, cooled down to 85 K. 
INTEGRAL is also equipped with two additional X-ray imagers, called the Joint European X-Ray Monitor, or JEM-X. The two imagers are identical, and use a gas scintillator detector to measure photons in the 3 - 35 keV region. Finally, INTEGRAL is also equipped with an optical telescope, called the Optical Monitoring Camera, OMC, to provide measurements in tandem with the other instrumentation.

\section{Coded Masks}
X-rays cannot be focused using conventional optical lenses, so the high energy instruments on-board INTEGRAL all make use of coded masks to form an image. Coded masks use a pattern of transparent and opaque elements in front of the sensor, and could be considered a sophisticated pinhole camera. Light entering a pinhole camera casts an image on the sensor, but greatly the pinhole greatly reduces the amount of light that can enter. Most coded masks have approximately two thirds of the mask opaque. As expected, light entering through a coded mask does not create an image, but creates a shadow based on the arrangement of the opaque and transparent elements. This shadow is the combination of all the images formed from each element. 

The design of a coded mask allows the usage of mathematical algorithms applied to the sensor output, to construct an image, a process called deconvolution. Furthermore the design of coded masks is such that a source in the instrument's field of view will cast a unique shadow based on the orientation to the instrument. This means that multiple sources in the field of view can be separated, at a very fine angular resolution for an instrument of its type. 
