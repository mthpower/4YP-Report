% Chapter 2

\chapter{Theoretical Background} % Chapter title

\label{ch:theoretical} % For referencing the chapter elsewhere, use \autoref{ch:introduction} 

%----------------------------------------------------------------------------------------

\section{X-ray Binaries}
X-ray astronomy was confined to the study of our own Sun until the discovery of the first extra-solar in 1962 by Giacconi et. al. An X-ray detector was mounted in the head of a rocket, and launched into a high parabolic arc. The experimental team had discovered what would later be designated Scorpius X-1. Initially this discovery, and the discovery of other similar X-ray sources was puzzling to astronomers, who could account for how these stars were very luminous in X-rays, yet dim in optical wavelengths. If these were stars similar to our own, they should be significantly brighter in optical telescopes. 

\section{Uhuru/SAS-A}
The Uhuru/SAS-A satellite was the first mission to offer the insight that led to the modern field of X-ray astronomy. Operating in the X-ray band, it detected eclipses in the lightcurves of two sources, Her X-1 and Cen X-3. Whilst both sources have their differences, this result indicated that the systems are binary. Furthermore the orbital period and profile of the eclipses suggested that the stars were orbiting very close together. This then implied that the two stars could be interacting, with the possibility of mass transfer between the two. Also, one of the objects had to be compact. In the case of Her X-1 and Cen X-3, the compact object is a neutron star. We now understand that mass is transferred from a donor star to the compact object, accreting onto it. The initial mystery of how these sources were so luminous in X-rays is solved when one considers the energy transfer that is possible as the accreting material exchanges it\textquoteright{}s gravitational potential energy for heat. 

\section{Mass Transfer and the Roche Lobe}
X-ray Binary stars are broadly divided into two categories, those of a high mass companion star, and those of a low mass companion star.

\subsection{High Mass X-ray Binaries}
High Mass X-ray Binaries, HXMBs, have their mass transfer from donor star to compact object driven by a powerful stellar wind. Typically the donor star is an O or B type. The compact object accretes by simple infall as it ploughs through the relatively dense stellar wind. HMXBs do have accretion disks, but tend to be smaller than those of low mass systems. 

\subsection{Low Mass X-ray Binaries}
Conversely, Low Mass X-ray Binaries, LMXBs, have donor stars that are not sufficiently large or luminous to drive a powerful stellar wind. Thus the compact object has to accrete by a different mechanism. Consider the Binary system in a co-rotating frame of reference, such that the two stars appear stationary. Taking account of both centrifugal force and gravitational potential, the Roche Lobe is the local minimum or potential well where matter is gravitationally bound. Some stars extent is greater than their Roche Lobe, which allows matter to be transferred to the compact object by flowing over the L1 Lagrangian point. LXMBs tend to exhibit larger accretion disks than HMXBs and, due to the dynamics of the mass transfer, form a \textquotedblleft{}figure of eight\textquotedblright{} shape. 

\section{Classes of HXMB}
This project focusses on a sample of HMXBs, which astrophysicists subdivide into three broad categories.

\subsection{SGXBs}
SGXBs, Supergiant X-ray Binaries have compact objects that spend their orbit in the strong stellar wind of an OB Supergiant. They are typically the most luminous subclass of HXMBs, often showing short drops in the X-ray flux observed where the compact object is eclipsed by its donor companion. Whilst the orbital eccentricity of the accreting object is lower than the other classes of HMXB, variations in the X-ray flux observed is affected by the orbital period of the accreting object, as it moves closer and further away from the supergiant.

\subsection{BeXBs}
Be star X-ray Binaries are, as the name suggests, associated with spectral type Be donor stars. Of significance to the behaviour of the X-ray flux is the fact that Be stars typically form equatorial disks of matter. This is thought to be aided by the tendency for Be stars to rotate rapidly. Furthermore, BeXB systems tend to have the neutron star in an eccentric orbit. As a result, the X-ray luminosity of these systems increases rapidly as the neutron star approaches it\textquoteright{}s periastron, when it intersects with this equatorial region of material surrounding the Be star, and is relatively quiet for the rest of its orbit. 

\subsection{SFXTs}
Finally, Superfast X-ray Transients are a class that share common features with both SGXBs and BeXBs. SFXTs show outbursts similar BeXBs, but on a shorter timescale, and in addition are associated with the OB supergiants. One proposed explanation to explain the extremely short lived outbursts of these sources compared to the classical and persistent SGXBs is that the wind of these stars is inhomogeneous, and that the outbursts are caused by sudden accretion onto the neutron star as it moves through a \textquotedblleft{}clumpy\textquotedblright{} patch of solar wind. This may not a be totally sufficient explanation, since SFXTs exhibit some eccentricity in the orbit of the neutron star, and it is thought that it may be a combination of these factors that leads to the large and short outbursts. 