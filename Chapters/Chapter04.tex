% Chapter 4

\chapter{Results} % Chapter title

\label{ch:results} % For referencing the chapter elsewhere, use \autoref{ch:introduction} 

%----------------------------------------------------------------------------------------
\section{List of Sources}
Twenty two sources were part of the sample; a full list is shown below. Results have been grouped according to suggested class of HMXB, and approximately in order of confidence.
\begin{itemize}
\item 1A0535+262
\item AXJ1910.7+0917
\item Cen X-3
\item EXO2030+375
\item IGRJ08408-4503
\item IGRJ11215-5952
\item IGRJ16418-4532
\item IGRJ16465-4507
\item IGRJ16479-4514
\item IGRJ17391-3021
\item IGRJ17407-2808
\item IGRJ17544-2619
\item IGRJ18027-2016
\item IGRJ18219-1347
\item IGRJ18410-0535
\item IGRJ18450-0435
\item IGRJ18483-0311
\item MAXIJ1409-619
\item OAO1657-415
\item SAXJ1818.6-1703
\item Vela X-1
\item XPer
\end{itemize}
\clearpage{}

\section{Supergiant X-ray Binaries}
\subsection{Cen X-3}

\begin{figure}[h!]
\centering
\includegraphics[width=130mm]{gfx/Fig7.png}
\caption{20 day binned lightcurve for Cen X-3.}
\label{Figure 7}
\end{figure}

Cen X-3 is a very bright and well studied source, and as it\textquoteright{}s designation would suggest, was one of the first X-ray sources to be discovered. It is a SGXB, and so is powered primarily by stellar wind loss. It has a well documented 2.087 day cycle.

\begin{figure}[h!]
\centering
\includegraphics[width=130mm]{gfx/Fig8.png}
\caption{Periodogram for Cen X-3. The expected peak at 2.087 days is very clear.}
\label{Figure 8}
\end{figure}

\ref{Figure 7} shows a binned lightcurve. Although this does not reveal anything useful, it is an example of a particularly good source. Cen X-3 is very luminous, and the error bars and count rates reflect this. \ref{Figure 8} shows a periodogram from Cen X-3. The expected peak at 2.087 days is very clear. Once folded on this period, \ref{Figure 9} shows the eclipse profile. From these results, it is easy to see that Cen X-3 shows the behaviour expected of a SGXB. It is luminous and active except when eclipsing, and provides a good example to measure other less well known and more uncertain sources against. 

\begin{figure}[h!]
\centering
\includegraphics[width=130mm]{gfx/Fig9.png}
\caption{Folded lightcurve for Cen X-3.}
\label{Figure 9}
\end{figure} 

\clearpage
\subsection{Vela X-1}

\begin{figure}[h!]
\centering
\includegraphics[width=130mm]{gfx/Fig10.png}
\caption{Periodogram for Vela X-1.}
\label{Figure 10}
\end{figure} 

Vela X-1 is another bright SGXB that was discovered early in the study of XRBs, and like Cen X-3 and Cyg X-1 could be considered a textbook example of it\textquoteright{}s class. The orbital period of the system is already known to be 8.964 days.

\ref{Figure 10} shows the periodogram, with a large peak at the expected 8.964 days. \ref{Figure 11} shows the folded light curve on that period, and the eclipse profile is shown convincingly. 

\begin{figure}[h!]
\centering
\includegraphics[width=130mm]{gfx/Fig11.png}
\caption{Folded lightcurve for Vela X-1. Period: 8.964 days.}
\label{Figure 11}
\end{figure} 

\subsection{OAO1657-415}
OAO1657-415 is very similar to both Vela X-1 and Cen X-3, and is another SGXB. It has a slightly longer known period of 10.4 days and it has been suggested that this source might not be entirely wind-fed, and that a proportion of the accretion may be through a disk. These results corroborate with the 10.4 day period, as shown in the periodogram \ref{Figure 12}, which peaks at 10.45 days. \ref{Figure 13} shows the folded lightcurve, and the eclipse, clearly showing that these results agree with the body of literature. 

\begin{figure}[h!]
\centering
\includegraphics[width=130mm]{gfx/Fig12.png}
\caption{Periodogram for OAO1657-415. Peak at 10.45 days.}
\label{Figure 12}
\end{figure} 

\begin{figure}[h!]
\centering
\includegraphics[width=130mm]{gfx/Fig13.png}
\caption{Folded lightcurve for OAO1657-415. Orbital period: 10.45 days.}
\label{Figure 13}
\end{figure} 

\subsection{IGRJ18027-2016}
IGRJ18027-2016 is another probable Supergiant X-ray Binary, and was the test source used in this project. The results for this source were shown in the previous section. \ref{Figure 3} shows the periodogram, and \ref{Figure 5} shows the folded light curve and eclipse, with a period of 4.570 days. 

\clearpage

\subsection{IGRJ16418-4532}

\begin{figure}[h!]
\centering
\includegraphics[width=130mm]{gfx/Fig14.png}
\caption{Periodogram for IGRJ16418-4532. Strongest peak at 3.74 days.}
\label{Figure 14}
\end{figure} 

IGRJ16418-4532 is a particularly interesting source. Two papers have been published, one by Romano et. al. (2012)\cite{Romano} and one by Sidoli et. al. (2012)\cite{Sidoli} that suggest that this source is an SFXT. Those papers made use of the Swift and XMM-Newton instruments respectively, and observed for around 40 ks or $\simeq3$ orbital periods each. The results in this report agree with the orbital period suggested, of 3.7 days, as shown in the periodogram, \ref{Figure 14}. However, when folded on this period, the source shows what appears to be an eclipse, see \ref{Figure 15}, with a greater \textquotedblleft{}on time\textquotedblright{} than \textquotedblleft{}off time\textquotedblright{}. 

\begin{figure}[h!]
\centering
\includegraphics[width=130mm]{gfx/Fig15.png}
\caption{Folded lightcurve for IGRJ16418-4532. Folded on 3.74 days. Note the apparent eclipse.}
\label{Figure 15}
\end{figure} 

There is likely to not be a discrepancy. The authors of the two papers describe some quasi-periodic flaring in their observations of IGRJ16418-4532. The source is not as bright as others in the sample, and may be that this behaviour is being averaged out in the folding process. The periodogram does not show such a strong signal for the peak folded on, however there are a number of interesting secondary peaks in the periodogram suggesting that there is consistent variability on other timescales. From the perspective of the methods and data in this project, IGRJ16418-4532 looks like a SGXB exhibiting an eclipse, however this is probably naive. 

\clearpage
\section{Superfast X-ray Transients}

\subsection{IGRJ18483-0311}

\begin{figure}[h!]
\centering
\includegraphics[width=130mm]{gfx/Fig16.png}
\caption{Periodogram for IGRJ16418-4532. Strongest peak at 18.57 days.}
\label{Figure 16}
\end{figure} 


IGRJ18483-0311 is one of the strongest SFXTs that this project\textquoteright{}s sample contains. \ref{Figure 16} shows the periodogram for the source, which exhibits a strong peak at 18.57 days. Furthermore, the folded lightcurve, shown in \ref{Figure 17} is quite well resolved. The source shows a periodic increase in flux, characteristic for SFXTs. The lightcurve shows some interesting fluctuations during the time when the source is on. It is uncertain why this is the case, the more likely option is that it is a false positive. 

\begin{figure}[h!]
\centering
\includegraphics[width=130mm]{gfx/Fig17.png}
\caption{Folded lightcurve for IGRJ16418-4532. Folded on 18.57 days.}
\label{Figure 17}
\end{figure} 

\clearpage

\subsection{IGRJ16479-4514}

\begin{figure}[h!]
\centering
\includegraphics[width=130mm]{gfx/Fig18.png}
\caption{Periodogram for IGRJ16479-4514. Strongest peak at 11.89 days.}
\label{Figure 18}
\end{figure} 

IGRJ16479-4514 is far more difficult to classify from the results presented. The source is most likely a transient source, since the strongest period is fairly short at 11.89 days. The periodogram, \ref{Figure 18}, is not especially clear, and so it\textquoteright{}s possible that if the period folded on is not precise, then this would have the effect of smearing out the lightcurve. This could be possible given the folded lightcurve, \ref{Figure 19}, which shows a broad and fluctuating peak where the source is on. Overall, the lightcurve does appear to show a fast rise in the flux, followed by a slower decline. 

\begin{figure}[h!]
\centering
\includegraphics[width=130mm]{gfx/Fig19.png}
\caption{Folded lightcurve for IGRJ16479-4514. Folded on 11.89 days. Note the fast rise, and slower decline.}
\label{Figure 19}
\end{figure} 

\subsection{IGRJ16465-4507}

\begin{figure}[h!]
\centering
\includegraphics[width=130mm]{gfx/Figure20.png}
\caption{Periodogram for IGRJ16465-4507. Strongest peak at 30.32 days.}
\label{Figure 20}
\end{figure} 

IGRJ16465-4507 is another example of a SFXT. The source is fairly dim, and the error bars on the lightcurve reflect this. \ref{Figure 20} shows the periodogram for the source, which has a strong peak at 30.32 days. Furthermore, the folded lightcurve, \ref{Figure 21} shows a profile that is indicative of a Superfast X-ray Transient. 

\begin{figure}[h!]
\centering
\includegraphics[width=130mm]{gfx/Figure21.png}
\caption{Folded lightcurve for IGRJ16465-4507. Folded on 30.32 days.}
\label{Figure 21}
\end{figure} 


% \clearpage
% \section{Be X-ray Binaries}

% \subsection{1A0535+262}

% \subsection{}

